%!TeX program=xelatex
%==============================常用宏包、环境==============================%
\documentclass[twocolumn,a4]{article}
\usepackage{xeCJK} % For Chinese characters
\usepackage{amsmath, amsthm}
\usepackage{listings,xcolor}
\usepackage{geometry} % 设置页边距
\usepackage{fontspec}
\usepackage{graphicx}
\usepackage{fancyhdr} % 自定义页眉页脚
\setsansfont{Consolas} % 设置英文字体
\setmonofont[Mapping={}]{Consolas} % 英文引号之类的正常显示,相当于设置英文字体
\geometry{left=1cm,right=1cm,top=2cm,bottom=0.5cm} % 页边距
\setlength{\columnsep}{30pt}
% \setlength\columnseprule{0.4pt} % 分割线
%==============================常用宏包、环境==============================%

%==============================页眉、页脚、代码格式设置==============================%
% 页眉、页脚设置
\pagestyle{fancy}
% \lhead{CUMTB}
\lhead{\CJKfamily{hei} XCPC's Bizarre Adventure}
\chead{}
% \rhead{Page \thepage}
\rhead{\CJKfamily{hei} 第 \thepage 页}
\lfoot{} 
\cfoot{}
\rfoot{}
\renewcommand{\headrulewidth}{0.4pt} 
\renewcommand{\footrulewidth}{0.4pt}

% 代码格式设置
\lstset{
    language    = c++,
    numbers     = left,
    numberstyle = \tiny,
    breaklines  = true,
    captionpos  = b,
    tabsize     = 4,
    frame       = shadowbox,
    columns     = fullflexible,
    commentstyle = \color[RGB]{0,128,0},
    keywordstyle = \color[RGB]{0,0,255},
    basicstyle   = \small\ttfamily,
    stringstyle  = \color[RGB]{148,0,209}\ttfamily,
    rulesepcolor = \color{red!20!green!20!blue!20},
    showstringspaces = false,
}
%==============================页眉、页脚、代码格式设置==============================%

%==============================标题和目录==============================%
\title{\CJKfamily{hei} \bfseries XCPC Code Library}
\author{XCPC's Bizarre Adventure}
\renewcommand{\today}{\number\year 年 \number\month 月 \number\day 日}

\begin{document}\small
%\begin{titlepage}
\maketitle
%\end{titlepage}

\pagestyle{empty}
\renewcommand{\contentsname}{目录}
\tableofcontents
%\newpage
%\clearpage
%\newpage
\pagestyle{fancy}
\setcounter{page}{1}   %new page
%==============================标题和目录==============================%

%==============================正文部分==============================%
\section{数据结构}
\subsection{可并堆}
\begin{lstlisting}
struct Heap {
	LL val[MAXN], mult[MAXN], plus[MAXN];
	int lc[MAXN], rc[MAXN];
	void Mult(int t, LL dt) { if (t) val[t] *= dt, plus[t] *= dt, mult[t] *= dt; }
	void Plus(int t, LL dt) { if (t) val[t] += dt, plus[t] += dt; }
	void pushdown(int t) {
		if (mult[t] != 1) Mult(lc[t], mult[t]), Mult(rc[t], mult[t]), mult[t] = 1;
		if (plus[t]) Plus(lc[t], plus[t]), Plus(rc[t], plus[t]), plus[t] = 0;
	}
	int merge(int u, int v) {
		if (!u || !v) return u ^ v;
		if (val[u] > val[v]) swap(u, v);
		pushdown(u), rc[u] = merge(rc[u], v), swap(lc[u], rc[u]);
		return u;
	}
	int pop(int u) {
		pushdown(u);
		int t = merge(lc[u], rc[u]);
		lc[u] = rc[u] = 0;
		return t;
	}
}heap;
\end{lstlisting}
\subsection{Splay}
\begin{lstlisting}
int sz[MAXN], va[MAXN], ch[MAXN][2], flag[MAXN], n, m, cnt, rt;//flag为翻转标记
void maintain(int o) { sz[o] = sz[ch[o][0]] + sz[ch[o][1]] + 1; }
void pushdown(int x) {
    if (flag[x]) {
        flag[x] = 0; swap(ch[x][0], ch[x][1]);
        flag[ch[x][0]] ^= 1; flag[ch[x][1]] ^= 1;
    }
}
int build(int n) {
    if (!n) return 0;
    int lc = build(n >> 1);
    int now = ++ cnt;
    va[now] = now - 1;
    ch[now][0] = lc;
    ch[now][1] = build(n - (n >> 1) - 1);
    maintain(now);
    return now;
}
void Init() {
    n = read(); m = read();
    rt = build(n + 1);
}
int cmp(int x, int k) {
    if (k == sz[ch[x][0]] + 1) return -1;
    return k > sz[ch[x][0]];
}
void rotate(int &o, int d) {
    int k = ch[o][d ^ 1];
    ch[o][d ^ 1] = ch[k][d]; ch[k][d] = o;
    maintain(o); maintain(k); o = k;
}
void splay(int &o, int k) {
    pushdown(o);
    int d = cmp(o, k);
    if (d == -1) return ;
    if (d) k -= sz[ch[o][0]] + 1;
    int p = ch[o][d];
    pushdown(p);
    int d2 = cmp(p, k);
    if (d2 >= 0) {
        int k2 = d2 ? k - sz[ch[p][0]] - 1 : k;
        splay(ch[p][d2], k2);
        if (d == d2) rotate(o, d ^ 1); else rotate(ch[o][d], d);
    }
    rotate(o, d ^ 1);
}
int merge(int x, int y) {
    splay(x, sz[x]);
    ch[x][1] = y, maintain(x);
    return x;
}
void split(int o, int k, int &l, int &r) {
    splay(o, k), l = o, r = ch[o][1];
    ch[l][1] = 0, maintain(l);
}

void Solve() {
    while (m --) {
        int l, r, le, ri, md, o;
        l = read(); r = read();
        split(rt, l, le, o);
        split(o, r - l + 1, md, ri);
        flag[md] ^= 1;
        rt = merge(merge(le, md), ri);
    }
}
\end{lstlisting}
\subsection{LCT}
\begin{lstlisting}
struct LCT {
    int fa[MAXN], ch[MAXN][2], rev[MAXN], xsum[MAXN];
    bool isrt(int x) { return x != ch[fa[x]][0] && x != ch[fa[x]][1]; }
    bool dir(int o) { return o != ch[fa[o]][0]; }
    void maintain(int o) { xsum[o] = xsum[ch[o][1]] ^ xsum[ch[o][0]] ^ w[o]; }
    void pushdown(int o) {
        if (rev[o]) rev[o] = 0; rev[ch[o][0]] ^= 1; rev[ch[o][1]] ^= 1; swap(ch[o][0], ch[o][1]);
    }
    void rotate(int o) {
        int f = fa[o], gf = fa[f], d = dir(o) ^ 1;
        fa[ch[o][d]] = f;
        ch[f][d ^ 1] = ch[o][d];
        fa[o] = gf;
        if (!isrt(f)) ch[gf][dir(f)] = o;
        fa[f] = o; ch[o][d] = f;
        maintain(f); maintain(o);
    }
    int sta[MAXN],top;
    void splay(int x) {
        sta[top = 1] = x;
        for (int t = x; !isrt(t); t = fa[t]) sta[++ top] = fa[t];
        while (top) pushdown(sta[top --]);
        for ( ; !isrt(x); rotate(x)) if(!isrt(fa[x])) rotate(dir(fa[x]) == dir(x) ? fa[x] : x);
    }
    void access(int o) { for (int t = 0; o; t = o, o = fa[o]) splay(o), ch[o][1] = t, maintain(o); }
    void makeroot(int x) { access(x); splay(x); rev[x] ^= 1; }
    void link(int x, int y) { makeroot(x); fa[x] = y; }
    void cut(int x, int y) {
        makeroot(x); access(y); splay(y);
        if (ch[y][0] == x) ch[y][0] = 0, fa[x] = 0, maintain(y);
    }
    int findroot(int x) {
        access(x); splay(x);
        while (ch[x][0]) x = ch[x][0];
        return x;
    }
}lct;
void Init() {
    n = read(); m = read();
    For(i, 1, n) w[i] = lct.xsum[i] = read();
}
void Solve() {
    while (m --) {
        int op, x, y;
        op = read(); x = read(); y = read();
        if (!op) {
            lct.makeroot(x); lct.access(y); lct.splay(y);
            printf("%d\n", lct.xsum[y]);
        }
        else if (op == 1) { if (lct.findroot(x) != lct.findroot(y)) lct.link(x, y); }
        else if (op == 2) lct.cut(x, y);
        else { lct.access(x); lct.splay(x); w[x] = y; lct.maintain(x); }
    }
}
\end{lstlisting}
\subsection{支配树}
\begin{lstlisting}
//洛谷模板题
void Dfs(int u) {
	id[dfn[u] = ++clk] = u;
	for (int v : G[u])
		if (!dfn[v])
			fa[v] = u, Dfs(v);
}
int find(int x) {
	if (f[x] == x) return x;
	int res = find(f[x]);
	if (dfn[sdom[ran[f[x]]]] < dfn[sdom[ran[x]]])
		ran[x] = ran[f[x]];
	return f[x] = res;
}
int main() {
	scanf("%d%d", &n, &m);
	for (int i = 1; i <= m; i++) {
		int u, v;
		scanf("%d%d", &u, &v);
		G[u].push_back(v);
		H[v].push_back(u);
	}
	Dfs(1);
	for (int i = 1; i <= n; i++)
		sdom[i] = f[i] = ran[i] = i;
	for (int i = clk; i > 1; i--) {
		int tmp = id[i];
		for (int v: H[tmp]) {
			if (!dfn[v])
				continue;
			find(v);
			if (dfn[sdom[ran[v]]] < dfn[sdom[tmp]])
				sdom[tmp] = sdom[ran[v]];
		}
		f[tmp] = fa[tmp];
		tr[sdom[tmp]].push_back(tmp);
		tmp = fa[tmp];
		for (int v: tr[tmp]) {
			find(v);
			if (tmp == sdom[ran[v]])
				idom[v] = tmp;
			else
				idom[v] = ran[v];
		}
		tr[tmp].clear();
	}
	for (int i = 2; i <= clk; i++) {
		int tmp = id[i];
		if (idom[tmp] ^ sdom[tmp])
			idom[tmp] = idom[idom[tmp]];
	}
	for (int i = clk; i > 1; i--)
		ans[idom[id[i]]] += ++ans[id[i]];
	ans[1]++;
	for (int i = 1; i <= n; i++)
		printf("%d ", ans[i]);
	return 0;
}
\end{lstlisting}

\section{数论}
\subsection{Miller-Rabin \& Pollard-Rho(含快速乘)}
\begin{lstlisting}
LL mult(LL a,LL b,LL p){
	LL d = (LL)floor(a * (LD)b / p + 0.5);
	LL ret = a * b - d * p;
	if (ret < 0) ret += p;
	return ret;
}
class MillerRabin {
    private:
        #define Pcnt 12
        const int P[Pcnt]={2,3,5,7,11,13,17,19,61,2333,4567,24251};
        LL fpm(LL x,LL y,LL X) {
            LL t=1;while(y) y&1&&(t=mult(t,x,X)),x=mult(x,x,X),y>>=1;
            return t;
        }
        int Check(LL x,int p) {
            if(!(x%p)||fpm(p%x,x-1,x)^1) return 0;
            LL k=x-1,t;
			while(!(k&1)) {
                if((t=fpm(p%x,k>>=1,x))^1&&t^(x-1)) return 0;
                if(!(t^(x-1))) return 1;
            }
			return 1;
        }
    public:
        int isP(LL x) {
            if(x<2) return false;
            for(int i=0;i^Pcnt;++i) {if(!(x^P[i])) return true;if(!Check(x,P[i])) return false;}
            return true;
        }
};
class PollardRho {
    private:
        #define Rand(x) (1LL*rand()*rand()%(x)+1)
        LL ans;
		MillerRabin MR;
        LL gcd(LL x,LL y) {return y?gcd(y,x%y):x;}
        LL Work(LL x,int y) {
            int t=0,k=1;
			LL v0=Rand(x-1),v=v0,d,s=1;
			for(;;) {
                if(v=(mult(v,v,x)+y)%x,s=mult(s,abs(v-v0),x),!(v^v0)||!s) return x;
                if(++t==k) {
					if((d=gcd(s,x))^1) return d;
					v0=v,k<<=1;
				}
            }
        }
        void Resolve(LL x,int t) {
            if (!(x^1)||x<=ans) return;
			if(MR.isP(x)) {
				if (ans < x) ans = x;
				return;
			}
            LL y=x;
			while((y=Work(x,t--))==x);
			while(!(x%y))x/=y;
			Resolve(x,t),Resolve(y,t);
        }
    public:
        PollardRho() {srand(1926);}
        LL GetMax(LL x) {return ans=0,Resolve(x,302627441),ans;}
}P;
\end{lstlisting}
\subsection{二次剩余}
\begin{lstlisting}
struct field2{
	int x, y, a, p;
	field2():x(0), y(0), a(0), p(0){}
	field2(int x,int y,int a,int p):x(x),y(y),a(a),p(p){}
	field2 operator * (const field2 &f)const{
		int retx=(1ll * x * f.x + 1ll * y * f.y % p * a) % p;
		int rety=(1ll * x * f.y + 1ll * y * f.x) % p;
		return field2(retx, rety, a, p);
	}
	field2 fpm(int exp) const {
		field2 ret(1, 0, a, p), aux = *this;
		for ( ; exp > 0; exp >>= 1){
			if (exp & 1){
				ret = ret * aux;
			}
			aux = aux * aux;
		}
		return ret;
	}
};
std::vector <int> remain2(int a, int p){
	if (!a || p == 2) return {a};
	if (fpm(a, p - 1 >> 1, p) != 1) return {};
	if (p == 3) return {1, 2};
	while (true){
		field2 f(randint(p-1) + 1, randint(p - 1) + 1, a, p);
		f = f.fpm(p - 1 >> 1);
		if (f.x) continue;
		int ret = fpm(f.y, p - 2, p);
		return {min(ret, p - ret), max(ret, p - ret)};
	}
}
\end{lstlisting}
\subsection{扩展欧几里得}
\begin{lstlisting}
void exgcd(LL a, LL b, LL &x, LL &y) {
	if (!b) x=1, y=0;
	else exgcd(b,a%b,y,x),y-=a/b*x;
}
\end{lstlisting}
\subsection{欧拉函数}
\begin{itemize}
\item 若$p$为素数,则$\varphi(p) = p - 1$

若$i \bmod p=0$,那么 $\varphi(i×p) = p×\varphi(i)$

若$i \bmod p$不等于0,那么 $\varphi(i×p) = (p-1)×\varphi(i)$

\item 欧拉函数是积性函数,即当a,b互质时,$\varphi(a×b) = \varphi(a)×\varphi(b)$

\item n为奇数时,$\varphi(2×a) = \varphi(a)$(原因:2n为偶数,偶数和偶数一定不互质,所以只有2n与小于它的奇数互素的情况,则恰好就等于n的欧拉函数值)

\item p为素数时,$\varphi(p^a) = p ^ a - p ^ {a - 1}$(原因:一共有$p^a$个数,由于$p$为质数,所以与$p^a$不互素即包含质因子$p$的数的个数为$(p^a)/p = p ^ {a - 1}$,总数减去不互素的数即为$\varphi(p ^ a) = p ^ a - p ^ {a - 1}$

\item 设$p_1\ldots p_k$为$n$的质因数分解,则 $\varphi(x)=x(1-\frac{1}{p_1})(1-\frac{1}{p_2})\cdots(1-\frac{1}{p_k})$

\item $\sum_{d|n}\varphi(d)=n$
(找到所有的$gcd(i, n)=j$, 发现满足$gcd(t,n)=\frac{n}{d} (d|n)$的$t$有$\varphi(d)个$,然后发现可以不重复不遗漏地覆盖到所有$gcd(i,n)=j$)

\item 若$n>2$,那么$\varphi(n)$是偶数

\item 欧拉定理:若$(a,n)=1$,则$a^{\varphi(n)}\equiv1\pmod n$

由欧拉定理得出另一个结论:设m是正整数,$(a,m)=1$,则:$x\equiv ba^{\varphi(m)-1}\pmod m$是同余方程$ax\equiv b\pmod m$的解

\item 扩展欧拉定理:$a^x \equiv a^{x\bmod \varphi(p)+\varphi(p)[x > \varphi(p)]} \pmod p$
\end{itemize}
\subsection{莫比乌斯反演}
$$F(n)=\sum_{d|n}f(d)\Rightarrow f(n)=\sum_{d|n}\mu (d)F(\frac{n}{d})$$
$$F(n)=\sum_{n|d}f(d)\Rightarrow f(n)=\sum_{n|d}\mu(\frac{d}{n})F(d)$$
\subsection{杜教筛}
\subsubsection{$\mu$}
求$M(n)=\sum_{i=1}^{n}\mu(i)$

因为有性质$\sum_{d|n}\mu(d)=[n=1]$,所以有:

$$1=\sum_{i=1}^{n}\sum_{d|i}\mu(d)=\sum_{t=1}^{n}\sum_{d=1}^{\lfloor \frac{n}{t}\rfloor}\mu(d)=\sum_{i=1}^{n}M(\lfloor \frac{n}{i}\rfloor)$$

所以:$M(n)=1-\sum_{i=2}^{n}M(\lfloor \frac{n}{i} \rfloor)$,整除分块即可。
\begin{lstlisting}
const int MAXN = 1000005, MOD = 1000007;
int mu[MAXN], Sum_mu[MAXN], prime[MAXN >> 1], cnt, np[MAXN], beg[MOD], nex[MOD], n, e;
LL n1, n2, v[MOD], w[MOD];
void add(int uu, LL vv, LL ww) { v[++ e] = vv, w[e] = ww, nex[e] = beg[uu], beg[uu] = e; }
void Get_mu() {
    mu[1] = 1;
    For(i, 2, n) {
        if (!np[i]) prime[++ cnt] = i, mu[i] = -1;
        for (int j = 1; j <= cnt && prime[j] * i <= n; ++ j) {
            np[i * prime[j]] = 1;
            if (!(i % prime[j])) {
                mu[i * prime[j]] = 0;
                break;
            } else mu[i * prime[j]] = -mu[i];
        }
    }
    For(i, 1, n) Sum_mu[i] = Sum_mu[i - 1] + mu[i];
}
LL Calc(LL x) {
    int tmp = x % MOD;
    if (x <= n) return Sum_mu[x];
    for (int i = beg[tmp]; i; i = nex[i]) if (v[i] == x) return w[i];
    LL Ans = 1;
    for (LL l = 2, r; l <= x; l = r + 1)
        r = x / (x / l), Ans -= (r - l + 1) * 1ll * Calc(x / l);
    add(tmp, x, Ans);
    return Ans;
}
int main() {
    scanf("%lld%lld", &n1, &n2), n = (int)ceil(sqrt(n2 * 1.0)) * 10;
    Get_mu();
    printf("%lld\n", Calc(n2) - Calc(n1 - 1));
    return 0;
}
\end{lstlisting}
\subsubsection{$\varphi$}
求$S(n)=\sum_{i=1}^{n}\varphi(i)$

性质:$\sum_{d|n}\varphi(d)=n$

$$\sum_{i=1}^{n}i=\sum_{i=1}^{n}\sum_{d|i}\varphi(d)=\sum_{t=1}^{n}\sum_{d=1}^{\lfloor\frac{n}{t}\rfloor}\varphi(d)=\sum_{i=1}^{n}S(\lfloor\frac{n}{i}\rfloor)$$

所以:$S(n)=\sum_{i=1}^n i-\sum_{i=2}^{n}S(\lfloor \frac{n}{i}\rfloor)$
\subsection{CRT及扩展}
\subsubsection{CRT}
$m_1,m_2,\dots$两两互质,$M=\prod m_i$

对于同余方程组:
$$
\begin{cases}
x\equiv c_1 \pmod {m_1} \\
x\equiv c_2 \pmod {m_2} \\
\dots
\end{cases}
$$
在模$M$意义下有唯一解。

令$M_i = M/m_i$,则解$x_0 \equiv \sum c_i\times M_i \times M_i^{-1} \pmod M$
($M_i^{-1}$指模$m_i$意义下的逆元,若$m_i$不是质数就只能用扩欧而不能用费马小定理求逆元)

\subsubsection{exCRT}
将同余方程写成不定方程的形式:

$x=c_1+m_1\times y_1$, $x=c_2+m_2\times y_2$

考虑合并以上两个方程。

易得:$c_1+m_1\times y_1=c_2+m_2\times y_2$

移项得:$m_1\times y_1-m_2\times y_2=c_2-c_1$

于是就可以用扩欧解决这个方程,求出$y_1$的最小正整数解并带入$x_0=c_1+m_1\times y_1$

然后就可以将两个方程合并为: $x\equiv x_0\pmod {\mathrm{lca}(m_1,m_2)}$

\subsection{Lucas定理及扩展}
\subsubsection{Lucas定理}
$$\binom{n}{m}\bmod p = \binom{n \bmod p}{m \bmod p} \times \binom{\lfloor n/p\rfloor}{\lfloor m/p\rfloor} \bmod p$$
\subsubsection{exLucas}
对于$C_n^m\bmod p$,我们可以令$p=\prod_{i=1}^{q}p_i^{k_i}$,列出方程组:

$ans\equiv c_1 \pmod {p_1^{k_1}}$

$ans\equiv c_2 \pmod {p_2^{k_2}}$

$\cdots$

$ans\equiv c_q \pmod {p_q^{k_q}}$

由于$p_1^{k_1} \cdots p_q^{k_q}$两两互质,所以可以直接用最基础的中国剩余定理合并。

接下来的问题是如何求出$c_1\cdots c_q$即$C_n^m\bmod p_i^{k_i}$

我们要先分别求出$n!\bmod p_i^{k_i}$,$m!\bmod p_i^{k_i}$,$(n-m)!\bmod p_i^{k_i}$的值,发现形式是差不多的,所以我们现在只研究$n!\bmod p_i^{k_i}$

举这个例子:

假设$n=22$,$p_i=3$,$k_i=2$

那么$n!=1\times 2 \times \cdots \times 22$

然后将其中是3的倍数的数提出来:

$原式=(1 \times 2 \times 4 \times 5 \times 7 \times 8 \times 10 \times 11 \times 13 \times 14 \times 16 \times 17 \times 19 \times 20 \times 22)\times 3^6 \times(1\times 2\times 3 \times 4 \times 5 \times 6 \times7)$

然后发现这个式子可以分成三部分:

1、$p_i^{k_i}$,这个可以直接快速幂

2、对于阶乘,我们可以递归求解

3、关键是怎么求第一部分的除去了3的倍数的数列的积

解决方法:

考虑将$1\dots n$分段,每$p_i^{k_i}$个数为一段,并去除可以被三整除的数,可以发现一个性质:

$$(1 \times 2 \times 4 \times 5 \times 7 \times 8) \equiv (10 \times 11 \times 13 \times 14 \times 16 \times 17)\pmod {p_i^{k_i}}$$

然后对于剩下的数,一定不超过$p_i^{k_i}$个,直接暴力求解即可。

另外,还有一个问题在计算除以$m!\pmod {p_i^{k_i}}$,$(n-m)!\pmod {p_i^{k_i}}$时,当然需要乘以其关于模数的乘法逆元,但是如果它们不与模数互质,就无法直接求出逆元了。所以我们需要先将数中质因子$p_i$除去,求出逆元后再乘上来。

(注:计算$n!$中质因子$p_i$的个数公式为:$x=\sum_{j=1}^{\infty}\lfloor \frac{n}{p_i^{j}} \rfloor$)

\begin{lstlisting}
LL fac(LL n, LL p, LL pk) {
	if (!n) return 1;
	LL res = 1;
	for (LL i = 2; i <= pk; ++ i)
		if (i % p) (res *= i) %= pk;
	res = fpm(res, n / pk, pk);
	for (LL i = 2; i <= n % pk; ++ i)
		if (i % p) (res *= i) %= pk;
	return res * fac(n / p, p, pk) % pk;
}
LL inv(LL n, LL Mod) {
	static LL x, y, t;
	gcd(n, Mod, x, y);
	t = ((x % Mod) + Mod) % Mod;
	return t;
}
LL C(LL n, LL m, LL p, LL k, LL pk) {
	if (n < m) return 0;
	LL t1 = fac(n, p, pk), t2 = fac(m, p, pk), t3 = fac(n - m, p, pk), cnt = 0;
	for (LL i = n; i; i /= p) cnt += i / p;
	for (LL i = m; i; i /= p) cnt -= i / p;
	for (LL i = n - m; i; i /= p) cnt -= i / p;
	return t1 * inv(t2, pk) % pk * inv(t3, pk) % pk * fpm(p, cnt, pk) % pk;
}
LL CRT(LL c, LL m) { return c * inv(p / m, m) % p * (p / m) % p; }
LL exLucas(LL n, LL m) {
	LL Ans = 0, tmp = p;
	for (int i = 2; i * i <= tmp; ++ i)
		if (!(tmp % i)) {
			LL cnt = 0, prod = 1;
			while (!(tmp % i)) tmp /= i, prod *= i, ++ cnt;
			(Ans += CRT(C(n, m, i, cnt, prod), prod)) %= p;
		}
	if (tmp > 1) (Ans += CRT(C(n, m, tmp, 1, tmp), tmp)) %= p;
	return Ans;
}
\end{lstlisting}
\subsection{原根}
对$\varphi(p)$进行质因数分解,若恒有$g^{\varphi(p)/p_i}\not \equiv 1 \pmod p$,则$g$为$p$的原根。


\section{组合数学}
\subsection{二项式反演}
$$
f(n) = \sum_{k=p}^n \binom{n}{k} g(k)
$$

$$
g(n) = \sum_{k=p}^n (-1)^{n-k} \binom{n}{k} f(k)
$$
\subsection{斯特林数}
\subsubsection{第一类斯特林数}
$$
{n \brack m} = {n-1 \brack m-1} + (n-1){n-1 \brack m}
$$
\subsubsection{第二类斯特林数}
递推:${n\brace m} = {n-1\brace m-1} + m {n-1 \brace m}$

容斥:$${n \brace m} = \frac{1}{m!} \sum_{k=0}^m (-1)^k \binom{m}{k} (m-k)^n$$

关于容斥的理解:枚举空盒子的个数,其它的随便乱放,由于盒子是相同的,所以要除以$m!$。

整理得到:
$$
{n \brace m} = \sum_{k=0}^m (-1)^k \times \frac{1}{k!} \times \frac{(m-k)^n}{(m-k)!}
$$
可以用NTT求解所有的${n\brace i}$。

重要性质:
$$
n^k = \sum_{i=0}^k {k \brace i} \binom{n}{i} i!
$$
理解:左边是将$k$个球放在$n$个盒子里;右边枚举非空盒子的个数,从$n$个盒子中选出$i$个,将$k$个球放在这$i$个盒子里,由于盒子是不同的,所有要乘$i!$。
这个式子还能写成:
$$
n^k = \sum_{i=1}^k {k \brace i} n^{\underline{i}}
$$

第二类斯特林数的展开式:
$$
{n \brace m} m! = \sum_{k=0}^m (-1)^k \binom{m}{k} (m-k)^n
$$
理解:左边是将$n$个数分成$m$个集合且集合有序的方案数;右边$k$枚举至少多少个集合是空集,然后在 $m$个集合中选$k$个成为空集,$n$个数乱放在剩下的集合中。
\subsubsection{斯特林数反演}
$$
f(n) = \sum_{i=1}^n {n \brace i} g(i)
$$
$$
g(n) = \sum_{i=1}^n (-1)^{n-i} {n \brack i} f(i)
$$
\subsection{其它}
\subsubsection{Matrix-Tree定理}
$G$的度数矩阵$D_G$是一个$n\times n$的矩阵,当$i\not = j$时,$D_{i,j} = 0$;$D_{i,i}$的值为节点度数。

$G$的邻接矩阵$A_G$也是一个$n\times n$的矩阵,当$i,j$直接相连时,$A_{i,j}=1$、否则为$0$。

我们定义Kirchhoff矩阵(也叫拉普拉斯算子)为$C_G = D_G - A_G$,则Matrix-Tree定理可描述为:图G的所有不同生成树的个数等于其Kirchhoff矩阵$C_G$任何一个$n-1$阶主子式的行列式的绝对值。(所谓$n-1$阶主子式,即对于$r$($1\le r\le n$),将$C_G$的第$r$行、第$r$列同时去掉后得到的新矩阵)
\subsubsection{Best定理}
对于一个有向图,其欧拉回路的个数等于以起点为根的树形图的个数乘以每个点度数(入度必须等于出度)减1的阶乘。

至于树形图个数,仍然可以用Kirchhoff矩阵计算:度数矩阵改为入度、$n-1$阶主子式只能去掉根的那一阶。

\subsubsection{错排公式}
f(x) = x(f(x-1)+f(x-2))
初始化:$f(0)=1,f(1)=0,f(2)=1$

\subsubsection{皮克定理}


\subsubsection{Catalan数}
$C_0 = 1$

$C_n = \sum_{i=0}^{n-1}C_i C_{n-i-1} = \binom{2n}{n}\frac{1}{n+1}=\frac{(2n)!}{(n+1)!n!}=\binom{2n}{n}-\binom{2n}{n-1}=\frac{4n-2}{n+1}C_{n-1} = \prod_{k=2}^n \frac{n+k}{k}$

\section{计算几何}
\begin{lstlisting}
#define PI 3.1415926535897932384626
const double EPS = 1e-8;
using namespace std;
 
#define Vector Point
 
#define ChongHe 0
#define NeiHan 1
#define NeiQie 2
#define XiangJiao 3
#define WaiQie 4
#define XiangLi 5
 
int dcmp(double x) { return fabs(x) < EPS ? 0 : (x < 0 ? -1 : 1); }
 
struct Point {
    double x, y;
 
    Point(const Point& rhs): x(rhs.x), y(rhs.y) { } //拷贝构造函数
    Point(double x = 0.0, double y = 0.0): x(x), y(y) { }   //构造函数
 
    friend istream& operator >> (istream& in, Point& P) { return in >> P.x >> P.y; }
    friend ostream& operator << (ostream& out, const Point& P) { return out << P.x << ' ' << P.y; }
 
    friend Vector operator + (const Vector& A, const Vector& B) { return Vector(A.x+B.x, A.y+B.y); }
    friend Vector operator - (const Point& A, const Point& B) { return Vector(A.x-B.x, A.y-B.y); }
    friend Vector operator * (const Vector& A, const double& p) { return Vector(A.x*p, A.y*p); }
    friend Vector operator / (const Vector& A, const double& p) { return Vector(A.x/p, A.y/p); }
    friend bool operator == (const Point& A, const Point& B) { return dcmp(A.x-B.x) == 0 && dcmp(A.y-B.y) == 0; }
    friend bool operator < (const Point& A, const Point& B) { return A.x < B.x || (A.x == B.x && A.y < B.y); }
 
    void in(void) { scanf("%lf%lf", &x, &y); }
    void out(void) { printf("%lf %lf", x, y); }
};
 
struct Line {
    Point P;    //直线上一点
    Vector dir; //方向向量(半平面交中该向量左侧表示相应的半平面)
    double ang; //极角,即从x正半轴旋转到向量dir所需要的角(弧度)
 
    Line() { }  //构造函数
    Line(const Line& L): P(L.P), dir(L.dir), ang(L.ang) { }
    Line(const Point& P, const Vector& dir): P(P), dir(dir) { ang = atan2(dir.y, dir.x); }
 
    bool operator < (const Line& L) const { //极角排序
        return ang < L.ang;
    }
 
    Point point(double t) { return P + dir*t; }
};
 
typedef vector<Point> Polygon;
 
struct Circle {
    Point c;    //圆心
    double r;   //半径
 
    Circle() { }
    Circle(const Circle& rhs): c(rhs.c), r(rhs.r) { }
    Circle(const Point& c, const double& r): c(c), r(r) { }
 
    Point point(double ang) const { return Point(c.x + cos(ang)*r, c.y + sin(ang)*r); } //圆心角所对应的点
    double area(void) const { return PI * r * r; }
};
 
double Dot(const Vector& A, const Vector& B) { return A.x*B.x + A.y*B.y; }  //点积
double Length(const Vector& A){ return sqrt(Dot(A, A)); }
double Angle(const Vector& A, const Vector& B) { return acos(Dot(A, B)/Length(A)/Length(B)); }  //向量夹角
double Cross(const Vector& A, const Vector& B) { return A.x*B.y - A.y*B.x; }    //叉积
double Area(const Point& A, const Point& B, const Point& C) { return fabs(Cross(B-A, C-A)); }
 
//三边构成三角形的判定
bool check_length(double a, double b, double c) {
    return dcmp(a+b-c) > 0 && dcmp(fabs(a-b)-c) < 0;
}
bool isTriangle(double a, double b, double c) {
    return check_length(a, b, c) && check_length(a, c, b) && check_length(b, c, a);
}
 
//平行四边形的判定(保证四边形顶点按顺序给出)
bool isParallelogram(Polygon p) {
    if (dcmp(Length(p[0]-p[1]) - Length(p[2]-p[3])) || dcmp(Length(p[0]-p[3]) - Length(p[2]-p[1]))) return false;
    Line a = Line(p[0], p[1]-p[0]);
    Line b = Line(p[1], p[2]-p[1]);
    Line c = Line(p[3], p[2]-p[3]);
    Line d = Line(p[0], p[3]-p[0]);
    return dcmp(a.ang - c.ang) == 0 && dcmp(b.ang - d.ang) == 0;
}
 
//梯形的判定
bool isTrapezium(Polygon p) {
    Line a = Line(p[0], p[1]-p[0]);
    Line b = Line(p[1], p[2]-p[1]);
    Line c = Line(p[3], p[2]-p[3]);
    Line d = Line(p[0], p[3]-p[0]);
    return (dcmp(a.ang - c.ang) == 0 && dcmp(b.ang - d.ang)) || (dcmp(a.ang - c.ang) && dcmp(b.ang - d.ang) == 0);
}
 
//菱形的判定
bool isRhombus(Polygon p) {
    if (!isParallelogram(p)) return false;
    return dcmp(Length(p[1]-p[0]) - Length(p[2]-p[1])) == 0;
}
 
//矩形的判定
bool isRectangle(Polygon p) {
    if (!isParallelogram(p)) return false;
    return dcmp(Length(p[2]-p[0]) - Length(p[3]-p[1])) == 0;
}
 
//正方形的判定
bool isSquare(Polygon p) {
    return isRectangle(p) && isRhombus(p);
}
 
//三点共线的判定
bool isCollinear(Point A, Point B, Point C) {
    return dcmp(Cross(B-A, C-B)) == 0;
}
 
//向量绕起点旋转
Vector Rotate(const Vector& A, const double& rad) { return Vector(A.x*cos(rad)-A.y*sin(rad), A.x*sin(rad)+A.y*cos(rad)); }
 
//向量的单位法线(调用前请确保A 不是零向量)
Vector Normal(const Vector& A) {
    double len = Length(A);
    return Vector(-A.y / len, A.x / len);
}
 
//两直线交点(用前确保两直线有唯一交点,当且仅当Cross(A.dir, B.dir)非0)
Point GetLineIntersection(const Line& A, const Line& B) {
    Vector u = A.P - B.P;
    double t = Cross(B.dir, u) / Cross(A.dir, B.dir);
    return A.P + A.dir*t;
}
 
//点到直线距离
double DistanceToLine(const Point& P, const Line& L) {
    Vector v1 = L.dir, v2 = P - L.P;
    return fabs(Cross(v1, v2)) / Length(v1);
}
 
//点到线段距离
double DistanceToSegment(const Point& P, const Point& A, const Point& B) {
    if (A == B) return Length(P - A);
    Vector v1 = B - A, v2 = P - A, v3 = P - B;
    if (dcmp(Dot(v1, v2)) < 0) return Length(v2);
    if (dcmp(Dot(v1, v3)) > 0) return Length(v3);
    return fabs(Cross(v1, v2)) / Length(v1);
}
 
//点在直线上的投影
Point GetLineProjection(const Point& P, const Line& L) { return L.P + L.dir*(Dot(L.dir, P - L.P)/Dot(L.dir, L.dir)); }
 
//点在线段上的判定
bool isOnSegment(const Point& P, const Point& A, const Point& B) {
    //若允许点与端点重合,可关闭下面的注释
    //if (P == A || P == B) return true;
    // return dcmp(Cross(A-P, B-P)) == 0 && dcmp(Dot(A-P, B-P)) < 0;
    return dcmp(Length(P-A) + Length(B-P) - Length(A-B)) == 0;
}
 
//线段相交判定
bool SegmentProperIntersection(const Point& a1, const Point& a2, const Point& b1, const Point& b2) {
    //若允许在端点处相交,可适当关闭下面的注释
    //if (isOnSegment(a1, b1, b2) || isOnSegment(a2, b1, b2) || isOnSegment(b1, a1, a2) || isOnSegment(b2, a1, a2)) return true;
    double c1 = Cross(a2-a1, b1-a1), c2 = Cross(a2-a1, b2-a1);
    double c3 = Cross(b2-b1, a1-b1), c4 = Cross(b2-b1, a2-b1);
    return dcmp(c1)*dcmp(c2) < 0 && dcmp(c3)*dcmp(c4) < 0;
}
 
//多边形的有向面积
double PolygonArea(Polygon po) {
    int n = po.size();
    double area = 0.0;
    for(int i = 1; i < n-1; i++) {
        area += Cross(po[i]-po[0], po[i+1]-po[0]);
    }
    return area * 0.5;
}
 
//点在多边形内的判定(多边形顶点需按逆时针排列)
bool isInPolygon(const Point& p, const Polygon& poly) {
    int n = poly.size();
    for(int i = 0; i < n; i++) {
        //若允许点在多边形边上,可关闭下行注释
        // if (isOnSegment(p, poly[(i+1)%n], poly[i])) return true;
        if (Cross(poly[(i+1)%n]-poly[i], p-poly[i]) < 0) return false;
    }
    return true;
}
 
//过定点作圆的切线
int getTangents(const Point& P, const Circle& C, std::vector<Line>& L) {
    Vector u = C.c - P;
    double dis = Length(u);
    if (dcmp(dis - C.r) < 0) return 0;
    if (dcmp(dis - C.r) == 0) {
        L.push_back(Line(P, Rotate(u, PI / 2.0)));
        return 1;
    }
    double ang = asin(C.r / dis);
    L.push_back(Line(P, Rotate(u, ang)));
    L.push_back(Line(P, Rotate(u, -ang)));
    return 2;
}
 
//直线和圆的交点
int GetLineCircleIntersection(Line& L, const Circle& C, vector<Point>& sol) {
    double t1, t2;
    double a = L.dir.x, b = L.P.x - C.c.x, c = L.dir.y, d = L.P.y - C.c.y;
    double e = a*a + c*c, f = 2.0*(a*b + c*d), g = b*b + d*d - C.r*C.r;
    double delta = f*f - 4*e*g; //判别式
    if (dcmp(delta) < 0) return 0;  //相离
    if (dcmp(delta) == 0) { //相切
      t1 = t2 = -f / (2 * e);
      sol.push_back(L.point(t1));
      return 1;
    }
    t1 = (-f - sqrt(delta)) / (2.0 * e); sol.push_back(L.point(t1));    // 相交
    t2 = (-f + sqrt(delta)) / (2.0 * e); sol.push_back(L.point(t2));
    return 2;
}
 
//两圆位置关系判定
int GetCircleLocationRelation(const Circle& A, const Circle& B) {
    double d = Length(A.c-B.c);
    double sum = A.r + B.r;
    double sub = fabs(A.r - B.r);
    if (dcmp(d) == 0) return dcmp(sub) != 0;
    if (dcmp(d - sum) > 0) return XiangLi;
    if (dcmp(d - sum) == 0) return WaiQie;
    if (dcmp(d - sub) > 0 && dcmp(d - sum) < 0) return INTERSECTING;
    if (dcmp(d - sub) == 0) return NeiQie;
    if (dcmp(d - sub) < 0) return NeiHan;
}
 
//两圆相交的面积
double GetCircleIntersectionArea(const Circle& A, const Circle& B) {
    int rel = GetCircleLocationRelation(A, B);
    if (rel < INTERSECTING) return min(A.area(), B.area());
    if (rel > INTERSECTING) return 0;
    double dis = Length(A.c - B.c);
    double ang1 = acos((A.r*A.r + dis*dis - B.r*B.r) / (2.0*A.r*dis));
    double ang2 = acos((B.r*B.r + dis*dis - A.r*A.r) / (2.0*B.r*dis));
    return ang1*A.r*A.r + ang2*B.r*B.r - A.r*dis*sin(ang1);
}
 
//凸包(Andrew算法)
//如果不希望在凸包的边上有输入点,把两个 <= 改成 <
//如果不介意点集被修改,可以改成传递引用
Polygon ConvexHull(vector<Point> p) {
    //预处理,删除重复点
    sort(p.begin(), p.end());
    p.erase(unique(p.begin(), p.end()), p.end());
    int n = p.size(), m = 0;
    Polygon res(n+1);
    for(int i = 0; i < n; i++) {
        while(m > 1 && Cross(res[m-1]-res[m-2], p[i]-res[m-2]) <= 0) m--;
        res[m++] = p[i];
    }
    int k = m;
    for(int i = n-2; i >= 0; i--) {
        while(m > k && Cross(res[m-1]-res[m-2], p[i]-res[m-2]) <= 0) m--;
        res[m++] = p[i];
    }
    m -= n > 1;
    res.resize(m);
    return res;
}
 
//点P在有向直线L左边的判定(线上不算)
bool isOnLeft(const Line& L, const Point& P) {
    return Cross(L.dir, P-L.P) > 0;
}
 
//半平面交主过程
//如果不介意点集被修改,可以改成传递引用
Polygon HalfPlaneIntersection(vector<Line> L) {
    int n = L.size();
    int head, rear;     //双端队列的第一个元素和最后一个元素的下标
    vector<Point> p(n); //p[i]为q[i]和q[i+1]的交点
    vector<Line> q(n);  //双端队列
    Polygon ans;
 
    sort(L.begin(), L.end());   //按极角排序
    q[head=rear=0] = L[0];  //双端队列初始化为只有一个半平面L[0]
    for(int i = 1; i < n; i++) {
        while(head < rear && !isOnLeft(L[i], p[rear-1])) rear--;
        while(head < rear && !isOnLeft(L[i], p[head])) head++;
        q[++rear] = L[i];
        if (fabs(Cross(q[rear].dir, q[rear-1].dir)) < EPS) {    //两向量平行且同向,取内侧的一个
            rear--;
            if (isOnLeft(q[rear], L[i].P)) q[rear] = L[i];
        }
        if (head < rear) p[rear-1] = GetLineIntersection(q[rear-1], q[rear]);
    }
    while(head < rear && !isOnLeft(q[head], p[rear-1])) rear--; //删除无用平面
    if (rear - head <= 1) return ans;   //空集
    p[rear] = GetLineIntersection(q[rear], q[head]);    //计算首尾两个半平面的交点
 
    for(int i = head; i <= rear; i++) { //从deque复制到输出中
        ans.push_back(p[i]);
    }
    return ans;
}
\end{lstlisting}

\section{图论}
\subsection{点双}
\begin{lstlisting}
void dfs(int u, int fa) {
	int chs = 0;
	dfn[u] = low[u] = ++ tim;
	for (int i = beg[u]; i; i = nex[i]) if (v[i] != fa)
	{
		tmp = mp(u, v[i]);
		if (!dfn[v[i]]) {
			stk.push(tmp), ++ chs;
			dfs(v[i], u), chkmin(low[u], low[v[i]]);
			if (low[v[i]] >= dfn[u])
			{
				iscut[u] = 1;
				++ bccs, bcc[bccs].clear();
				for ( ; ; ) {
					tmp = stk.top(), stk.pop();
					if (co[tmp.x] != bccs) co[tmp.x] = bccs, bcc[bccs].pb(tmp.x);
					if (co[tmp.y] != bccs) co[tmp.y] = bccs, bcc[bccs].pb(tmp.y);
					if (u == tmp.x && v[i] == tmp.y) break;
				}
			}
		} else if (dfn[v[i]] < dfn[u])
			stk.push(tmp), chkmin(low[u], dfn[v[i]]);
	}
	if (!fa && chs == 1) iscut[u] = 0;
}
\end{lstlisting}
\subsection{边双}
\begin{lstlisting}
int dfn[MAXN], low[MAXN], clk, stk[MAXN], top, co[MAXN], cnt;
void pop(){ int u=stk[top--]; co[u]=cnt; }
void Tarjan(int u, int pa) {
    dfn[u] = low[u] = ++ clk, stk[++ top] = u;
    for (int i = beg[u]; i; i = nex[i]) if ((i >> 1) != pa) {
        if (!dfn[v[i]]) Tarjan(v[i], i >> 1), low[u] = min(low[u], low[v[i]]);
        else low[u] = min(low[u], low[v[i]]);
    }
    if (dfn[u] == low[u]) for (++ cnt; co[stk[top]] = cnt, stk[top --] != u; );
}
\end{lstlisting}
\subsection{虚树}
\begin{lstlisting}
bool cmp(const int& a, const int& b) { return dfn[a] < dfn[b]; }
//每次建树前记得清零
For(i, 1, tot) iskey[s[i] = read()] = 1;
if (!iskey[1]) s[++ tot] = 1;
sort(s + 1, s + 1 + tot, cmp);
stk[top = 1] = 1, e_ = 0;
for (int i = 2; i <= tot; ++ i)
{
    int u = s[i], lca = LCA(u, stk[top]);
    if (lca != stk[top])
    {
        while (top > 1 && dep[stk[top - 1]] >= dep[lca])
            add_(stk[top - 1], stk[top]), -- top;
        if (stk[top] != lca) add_(lca, stk[top]), stk[top] = lca;
    }
    stk[++ top] = u;
}
Fordown(i, top, 2) add_(stk[i - 1], stk[i]);
\end{lstlisting}
\subsection{仙人掌圆方树}
\begin{lstlisting}
//[BZOJ2125]求仙人掌上的最短路
void add1(int uu, int vv, int ww) { v1[++ e1] = vv, w1[e1] = ww, nex1[e1] = beg1[uu], beg1[uu] = e1; }
void add2(int uu, int vv, LL ww) { v2[++ e2] = vv, w2[e2] = ww, nex2[e2] = beg2[uu], beg2[uu] = e2; }
void Build(int u, int anc, LL dsum) {
    static LL d; ++ tot;
    for (int t = u; t != fa1[anc]; t = fa1[t])
        cir[t] = dsum, d = min(dis1[t] - dis1[anc], dsum - dis1[t] + dis1[anc]), add2(tot, t, d), add2(t, tot, d);
}
void DFS(int u, int fe) {
    dfn[u] = low[u] = ++ clk;
    for (int i = beg1[u]; i; i = nex1[i]) if ((i >> 1) != fe) {
        if (!dfn[v1[i]]) dis1[v1[i]] = dis1[u] + w1[i], fa1[v1[i]] = u, DFS(v1[i], i >> 1), chkmin(low[u], low[v1[i]]);
        else chkmin(low[u], dfn[v1[i]]);
        if (dfn[u] < low[v1[i]]) add2(u, v1[i], w1[i]), add2(v1[i], u, w1[i]);
    }
    for (int i = beg1[u]; i; i = nex1[i])
        if (fa1[v1[i]] != u && dfn[v1[i]] > dfn[u]) Build(v1[i], u, w1[i] + dis1[v1[i]] - dis1[u]);
}
void DFS(int u) {
    for (int i = beg2[u]; i; i = nex2[i]) if (v2[i] != fa2[0][u])
        fa2[0][v2[i]] = u, dis2[v2[i]] = dis2[u] + w2[i], dep[v2[i]] = dep[u] + 1, DFS(v2[i]);
}
int LCA(int u, int v) {
    if (dep[u] < dep[v]) swap(u, v);
    Fordown(i, 14, 0) if (dep[v] + (1 << i) <= dep[u]) u = fa2[i][u];
    if (u == v) return u;
    Fordown(i, 14, 0) if (fa2[i][u] != fa2[i][v]) u = fa2[i][u], v = fa2[i][v];
    return fa2[0][u];
}
int climb(int u, int anc) {
    Fordown(i, 14, 0) if (dep[fa2[i][u]] > dep[anc]) u = fa2[i][u];
    return u;
}
LL dist(int u, int v) {
    static LL t; t = llabs(dis1[u] - dis1[v]);
    assert(cir[u] == cir[v]);
    return min(t, cir[u] - t);
}
int main() {
    static int m, q, uu, vv, ww, lca, ua, va;
    tot = n = read(), m = read(), q = read();
    while (m --) uu = read(), vv = read(), ww = read(), add1(uu, vv, ww), add1(vv, uu, ww);
    DFS(1, 0), DFS(1);
    For(j, 1, 14) For(i, 1, tot) fa2[j][i] = fa2[j - 1][fa2[j - 1][i]];
    while (q --) {
        uu = read(), vv = read(), lca = LCA(uu, vv);
        if (lca <= n) printf("%lld\n", dis2[uu] + dis2[vv] - (dis2[lca] << 1));
        else ua = climb(uu, lca), va = climb(vv, lca), printf("%lld\n", dis2[uu] + dis2[vv] - dis2[ua] - dis2[va] + dist(ua, va));
    }
    return 0;
}
\end{lstlisting}
\subsection{一般图圆方树}
\begin{lstlisting}
//[APIO2018]铁人两项
int n, e = 1, beg1[maxn], beg2[maxn << 1], nex[maxm << 1], v[maxm << 1], sz[maxn << 2], dfn[maxn], low[maxn], clk, w[maxn << 1], tot, all, stk[maxn], top, cnt;
LL  Ans;
void add1(int uu, int vv) { v[++ e] = vv, nex[e] = beg1[uu], beg1[uu] = e; }
void add2(int uu, int vv) { v[++ e] = vv, nex[e] = beg2[uu], beg2[uu] = e; }
void DFS(int u, int fe) {
    dfn[u] = low[u] = ++ clk, stk[++ top] = u, w[u] = -1;
    for (int i = beg1[u]; i; i = nex[i]) if ((i >> 1) != fe) {
        if (!dfn[v[i]]) {
            DFS(v[i], i >> 1), chkmin(low[u], low[v[i]]);
            if (low[v[i]] >= dfn[u]) {
                add2(u, ++ tot), cnt = 1;
                do add2(tot, stk[top]), ++ cnt; while (stk[top --] != v[i]);
                w[tot] = cnt;
            }
        } else chkmin(low[u], dfn[v[i]]);
    }
}
void getsz(int u) {
    sz[u] = u <= n;
    for (int i = beg2[u]; i; i = nex[i])
        getsz(v[i]), sz[u] += sz[v[i]];
}
void DP(int u) {
    int pre = u <= n;
    for (int i = beg2[u]; i; i = nex[i])
        DP(v[i]), Ans += (LL)pre * w[u] * sz[v[i]], pre += sz[v[i]];
    Ans += (LL)sz[u] * (all - sz[u]) * w[u];
}
int main() {
    static int m, uu, vv;
    tot = n = read(), m = read();
    while (m --) uu = read(), vv = read(), add1(uu, vv), add1(vv, uu);
    For(i, 1, n) if (!dfn[i]) DFS(i, 0), getsz(i), all = sz[i], DP(i);
    printf("%lld\n", Ans << 1);
    return 0;
}
\end{lstlisting}
\subsection{网络流}
\begin{lstlisting}
namespace MF {
    int e = 1, f[MAXM << 1], v[MAXM << 1], beg[MAXN], nex[MAXM << 1], S, T;
    void add(int uu, int vv, int ff) {
        v[++ e] = vv, f[e] = ff, nex[e] = beg[uu], beg[uu] = e;
        v[++ e] = uu, f[e] = 0, nex[e] = beg[vv], beg[vv] = e;
    }
    void init() {
        S = n + 1, T = n + 2;
		//add edges...
    }
    int lev[MAXN], beg1[MAXN];
    bool BFS() {
        static queue<int> q;
        memset(lev, -1, sizeof lev);
        while (!q.empty()) q.pop();
        for (lev[S] = 0, q.push(S); !q.empty(); q.pop()) {
            int u = q.front();
            for (int i = beg[u]; i; i = nex[i])
                if (f[i] && lev[v[i]] == -1) {
                    lev[v[i]] = lev[u] + 1, q.push(v[i]);
                }
        }
        return lev[T] != -1;
    }
    int DFS(int u, int flow) {
        if (u == T) return flow;
        int res = flow;
        for (int &i = beg1[u]; i; i = nex[i]) {
            if (lev[v[i]] == lev[u] + 1 && f[i]) {
                int t = DFS(v[i], min(res, f[i]));
                f[i] -= t, f[i ^ 1] += t;
                if (!(res -= t)) return flow;
            }
        }
        return flow - res;
    }
    int main() {
        int FLOW = 0;
        while (BFS()) memcpy(beg1, beg, sizeof beg), FLOW += DFS(S, 2);
		return FLOW;
    }
}
\end{lstlisting}
\subsection{费用流}
\begin{lstlisting}
//记得反向流是负边权
int BFS() {
    static deque<int> q;
    For(i, 1, n) dis[i] = INF; Set(vis, 0);
    dis[s] = 0; vis[s] = 1; q.pb(s);
    while (!q.empty()) {
        int u = q.front(); q.pop_front(); vis[u] = 0;
        for (int i = beg[u]; i; i = nex[i])
            if (f[i] && chkmin(dis[v[i]], dis[u] + w[i]))
                if (!vis[v[i]]) {
                    vis[v[i]] = 1;
                    if (!q.empty() && dis[q.front()] > dis[v[i]]) q.pf(v[i]); else q.pb(v[i]);
                }
    }
    return dis[t] != INF;
}
int DFS(int u, int flow) {
    if (u == t) return flow;
    vis[u] = 1;
    int res = flow, tmp;
    for (int i = beg[u]; i; i = nex[i]) {
        if (vis[v[i]] || !f[i] || dis[v[i]] != dis[u] + w[i]) continue;
        tmp = DFS(v[i], min(f[i], res));
        f[i] -= tmp; f[i ^ 1] += tmp; Cost += tmp * w[i];
        if (!(res -= tmp)) return flow;
    }
    return flow - res;
}
void MCMF() {
    while (BFS()) {
        vis[t] = 1;
        while (vis[t]) Set(vis, 0), Flow += DFS(s, INF);
    }
}
\end{lstlisting}
\subsection{匈牙利算法}
\begin{lstlisting}
int DFS(int u){
	For(i,1,m)if(G[u][i]&&!vis[i]){
		vis[i]=1;
		if(!mat[i]||DFS(mat[i])){mat[i]=u;return 1;}
	}
	return 0;
}
For(i,1,n){
	memset(vis,0,sizeof(vis));
	if(DFS(i))++ans;
}
\end{lstlisting}
\subsection{带花树}
\begin{lstlisting}
int n, m, v[maxm << 1], e, nex[maxm << 1], beg[maxn], clk, fa[maxn], pre[maxn], mat[maxn], Ans, tim[maxn], vis[maxn];
queue<int> q;
void add(int uu, int vv) { v[++ e] = vv, nex[e] = beg[uu], beg[uu] = e; }
int find(int x) { return fa[x] == x ? x : fa[x] = find(fa[x]); }
int LCA(int u, int v) {
	for (++ clk;; swap(u, v)) if (u) {
		u = find(u);
		if (tim[u] == clk) return u;
		tim[u] = clk, u = pre[mat[u]];
	}
}
void blossom(int u, int v, int lca) {
	while (find(u) != lca) {
		pre[u] = v, v = mat[u];
		if (vis[v] == 2) vis[v] = 1, q.push(v);
		if (find(u) == u) fa[u] = lca;
		if (find(v) == v) fa[v] = lca;
		u = pre[v];
	}
}
int BFS(int s) {
	For(i, 1, n) fa[i] = i;
	Set(vis, 0), Set(pre, 0);
	while (!q.empty()) q.pop();
	q.push(s), vis[s] = 1;
	while (!q.empty()) {
		int u = q.front();
		q.pop();
		for (int i = beg[u]; i; i = nex[i]) {
			if (find(u) == find(v[i]) || vis[v[i]] == 2) continue;
			if (!vis[v[i]]) {
				vis[v[i]] = 2, pre[v[i]] = u;
				if (!mat[v[i]]) {
					for (int t = v[i], las; t; t = las)
						las = mat[pre[t]], mat[t] = pre[t], mat[pre[t]] = t;
					return 1;
				}
				vis[mat[v[i]]] = 1, q.push(mat[v[i]]);
			} else {
				int lca = LCA(u, v[i]);
				blossom(u, v[i], lca), blossom(v[i], u, lca);
			}
		}
	}
	return 0;
}
For(i, 1, n) if (!mat[i]) Ans += BFS(i);
\end{lstlisting}

\section{字符串}
\subsection{KMP}
\begin{lstlisting}
void getNext() {
	nex[0] = 0;
	For(i, 1, lent - 1) {
		int j = nex[i - 1] - 1;
		while (~j && T[j + 1] != T[i]) j = nex[j] - 1;
		if (T[j + 1] == T[i]) nex[i] = j + 2;
		else nex[i] = 0;
	}
}
void getPos() {
	int j = -1;
	Rep(i, lens) {
		while (~j && T[j + 1] != S[i]) j = nex[j] - 1;
		if (T[j + 1] == S[i]) {
			++ j;
			if (j == lent - 1)
				printf("%d\n", i - lent + 2), j = nex[j] - 1;
		}
	}
}
\end{lstlisting}
\subsection{AC自动机}
\begin{lstlisting}
struct Trie {
	int ids, ch[MAXN][26], fail[MAXN], cnt[MAXN], dep[MAXN];
	vector<int> id[MAXN];
	Trie() { ids = 1; }
	void insert(char *str, int nid) {
		int len = strlen(str), u = 1;
		Rep(i, len) {
			int c = str[i] - 97;
			if (ch[u][c]) u = ch[u][c];
			else u = ch[u][c] = ++ ids;
		}
		id[u].PB(nid);
	}
	void init() {
		static queue<int> q;
		Rep(i, 26) if (ch[1][i]) {
			fail[ch[1][i]] = 1;
			dep[ch[1][i]] = 1;
			q.push(ch[1][i]);
		}
		for (; !q.empty(); q.pop()) {
			int u = q.front();
			Rep(i, 26) {
				int v = ch[u][i];
				if (!v) continue;
				fail[v] = 1, dep[v] = 1;
				for (int w = fail[u]; w; w = fail[w])
					if (ch[w][i]) {
						fail[v] = ch[w][i];
						dep[v] = dep[fail[v]] + 1;
						break;
					}
				q.push(v);
			}
		}
	}
	void query(char *str) {
		int len = strlen(str), u = 1;
		static int ans[MAXN], bkt[MAXN], p[MAXN];
		Rep(i, len) {
			int c = str[i] - 97;
			while (u > 1 && !ch[u][c])
				u = fail[u];
			if (ch[u][c]) u = ch[u][c];
			else u = 1;
			++ cnt[u];
		}
		For(i, 1, ids) ++ bkt[dep[i]];
		For(i, 1, ids) bkt[i] += bkt[i - 1];
		For(i, 1, ids) p[bkt[dep[i]] --] = i;
		Fordown(i, ids, 1) {
			int u = p[i];
			cnt[fail[u]] += cnt[u];
			Rep(j, SZ(id[u]))
				ans[id[u][j]] = cnt[u];
		}
		For(i, 1, n)
			printf("%d\n", ans[i]);
	}
}trie;
\end{lstlisting}

\subsection{SA}
\begin{lstlisting}
namespace SA {
	int rk[MAXN << 1], tp[MAXN << 1], sa[MAXN], height[MAXN], m;
	void rsort(int n) {
		static int c[MAXN];
		For(i, 1, m) c[i] = 0;
		For(i, 1, n) ++ c[rk[i]];
		For(i, 1, m) c[i] += c[i - 1];
		Fordown(i, n, 1) sa[c[rk[tp[i]]] --] = tp[i];
	}
	void init(char *s, int n) {
		m = 26;
		For(i, 1, n) rk[i] = s[i] - 96, tp[i] = i;
		rsort(n);
		for (int k = 1; ; k <<= 1) {
			int p = 0;
			For(i, n - k + 1, n) tp[++ p] = i;
			For(i, 1, n) if (sa[i] > k) tp[++ p] = sa[i] - k;
			rsort(n), swap(tp, rk);
			rk[sa[1]] = m = 1;
			For(i, 2, n)
				rk[sa[i]] = tp[sa[i]] == tp[sa[i - 1]] && tp[sa[i] + k] == tp[sa[i - 1] + k] ? m : ++ m;
			if (m == n) break;
		}

		for (int i = 1, j, k = 0; i <= n; height[rk[i ++]] = k)
			for (k = k ? k - 1 : 0, j = sa[rk[i] - 1]; s[j + k] == s[i + k]; ++ k);

		For(i, 1, n) printf("%d ", sa[i]);
		putchar('\n');
		For(i, 2, n) printf("%d ", height[i]);
		putchar('\n');
	}
}
\end{lstlisting}

\subsection{SAM}
\begin{lstlisting}
int tot = 1, las = 1, fa[MAXN << 1], ch[MAXN << 1][26], sz[MAXN << 1], len[MAXN << 1], p[MAXN << 1], bkt[MAXN << 1];
void extend(int c) {
	int np = ++ tot, p = las;
	len[las = np] = len[p] + 1, sz[np] = 1;
	while (p && !ch[p][c]) ch[p][c] = np, p = fa[p];
	if (!p) fa[np] = 1;
	else {
		int q = ch[p][c];
		if (len[q] == len[p] + 1) fa[np] = q;
		else {
			int nq = ++ tot;
			Cpy(ch[nq], ch[q]), fa[nq] = fa[q], len[nq] = len[p] + 1;
			fa[q] = fa[np] = nq;
			while (p && ch[p][c] == q) ch[p][c] = nq, p = fa[p];
		}
	}
}
For(i, 1, tot) ++ bkt[len[i]];
For(i, 1, tot) bkt[i] += bkt[i - 1];
For(i, 1, tot) p[bkt[len[i]] --] = i;
\end{lstlisting}

\subsection{Manacher}
\begin{lstlisting}
n = read(), scanf("%s", s_ + 1);
s[++ len] = '#';
For(i, 1, n) s[++ len] = '$', s[++ len] = s_[i];
s[++ len] = '$', s[++ len] = '!';
For(i, 1, len) {
    if (s[i] != '$') continue;
    p[i] = i <= mx ? min(mx - i, p[(id << 1) - i]) : 0;
    while (s[i - p[i] - 1] == s[i + p[i] + 1]) ++ p[i];
    if (chkmax(mx, i + p[i])) id = i;
    if (!(p[i] & 1)) Ans += p[i] >> 1;
}
\end{lstlisting}

\subsection{PAM}
\begin{lstlisting}
//[HDU5421] 双端插入PAM,输出回文串个数和本质不同回文串个数
void init() {
    fa[1] = fa[0] = 1, Set(ch, 0), len[tot = 1] = -1, ans = 0, l = (r = 1e5) + 1, suf = pre = 0, Set(s, 0);
}
void extend(int i, int &las, int ty) {
    int p = las, c = (s[i] = getchar()) - 97;
    while (s[i] != s[i - len[p] * ty - ty]) p = fa[p];
    if (!ch[p][c]) {
        int np = ++ tot, k = fa[p];
        while (s[i] != s[i - len[k] * ty - ty]) k = fa[k];
        len[np] = len[p] + 2, dep[np] = dep[fa[np] = ch[k][c]] + 1, ch[p][c] = np;
    }
    ans += dep[las = ch[p][c]];
    if (len[las] == r - l + 1) pre = suf = las;
}
int main() {
    static int T, opt;
    while (~scanf("%d", &T)) {
        init();
        while (T --) {
            opt = read();
            if (opt < 3) opt == 1 ? extend(--l, pre, -1) : extend(++ r, suf, 1);
            else opt == 3 ? printf("%d\n", tot - 1) : printf("%lld\n", ans);
        }
    }
    return 0;
}
\end{lstlisting}

\section{多项式}
\subsection{多项式全家桶}
\begin{lstlisting}
const int MAXN = 1 << 19, MOD = 998244353, g0 = 3;
int ig0;
int pw[MAXN], pw_[MAXN];
int fac[MAXN], ifac[MAXN];
int fpm(int a, int b = MOD - 2) {
    int ans = 1;
    for (; b; b >>= 1, a = (LL)a * a % MOD)
        if (b & 1)
            ans = (LL)ans * a % MOD;
    return ans;
}
int ad(int x, int y) { return (x += y) >= MOD ? x - MOD : x; }
void inc(int &x, int y) { if ((x += y) >= MOD) x -= MOD; }
int times2(int x) { return (x += x) >= MOD ? x - MOD : x; }
int Init(int n) {
    int pt, N;
    for (pt = 0, N = 1; N <= n; N <<= 1, ++ pt);
    ig0 = fpm(g0, MOD - 2);
    For(i, 1, pt + 1)
    pw[1 << i] = fpm(g0, (MOD - 1) / (1 << i));
	pw_[1 << i] = fpm(ig0, (MOD - 1) / (1 << i));
    fac[0] = 1;
    For(i, 1, N - 1) fac[i] = (LL)fac[i - 1] * i % MOD;
    ifac[N - 1] = fpm(fac[N - 1]);
    Fordown(i, N - 1, 1) ifac[i - 1] = (LL)ifac[i] * i % MOD;
    return N;
}
void NTT(int *a, int n, int ty) {
    static int rev[MAXN];
    static int W[MAXN];
    int pt = __builtin_ctz(n);
    Rep(i, n) if (i < (rev[i] = ((rev[i >> 1] >> 1) | ((i & 1) << (pt - 1))))) swap(a[i], a[rev[i]]);
    for (int i = 2, i2 = 1; i <= n; i2 = i, i <<= 1) {
        W[0] = 1, W[1] = ty > 0 ? pw[i] : pw_[i];
        For(j, 2, i2 - 1) W[j] = (LL)W[j - 1] * W[1] % MOD;
        for (int j = 0; j < n; j += i) {
            Rep(k, i2) {
                int x = a[j + k], y = (LL)a[j + k + i2] * W[k] % MOD;
                a[j + k] = ad(x, y), a[j + k + i2] = ad(x, MOD - y);
            }
        }
    }
    if (ty < 1) {
        int inv = fpm(n);
        Rep(i, n) a[i] = (LL)a[i] * inv % MOD;
    }
}
void Mult(int *f, int *g, int n, int *h) {
    static int f_[MAXN], g_[MAXN];
    Rep(i, n) f_[i] = f[i], g_[i] = g[i];
    For(i, n, n * 2 - 1) f_[i] = g_[i] = 0;
    NTT(f_, n << 1, 1), NTT(g_, n << 1, 1);
    Rep(i, n << 1) h[i] = (LL)f_[i] * g_[i] % MOD;
    NTT(h, n << 1, -1);
}
void Mult(int *f1, int *f2, int *f3, int n, int *h) {
    static int f1_[MAXN], f2_[MAXN], f3_[MAXN];
    Rep(i, n) f1_[i] = f1[i], f2_[i] = f2[i], f3_[i] = f3[i];
    For(i, n, n * 2 - 1) f1_[i] = f2_[i] = f3_[i] = 0;
    NTT(f1_, n << 1, 1), NTT(f2_, n << 1, 1), NTT(f3_, n << 1, 1);
    Rep(i, n << 1) h[i] = (LL)f1_[i] * f2_[i] % MOD * f3_[i] % MOD;
    NTT(h, n << 1, -1);
}
namespace Inv {
static int f[MAXN];
void Inv_(int *g, int n) {
    static int h[MAXN];
    if (n == 1) {
        g[0] = fpm(f[0]);
        return;
    }
    Inv_(g, n >> 1);
    Mult(g, g, f, n, h);
    Rep(i, n) g[i] = ad(ad(g[i], g[i]), MOD - h[i]);
}
void Inv(int *A, int n, int *ans) {
    Rep(i, n) f[i] = A[i], ans[i] = 0;
    Inv_(ans, n);
}
}
void Int(int *f, int n, int *g) {
    Fordown(i, n - 1, 1) g[i] = (LL)f[i - 1] * fpm(i) % MOD;
    g[0] = 0;
}
void Der(int *f, int n, int *g) {
    For(i, 1, n - 1) g[i - 1] = (LL)f[i] * i % MOD;
    g[n - 1] = 0;
}
void Ln(int *f, int n, int *g) {
    static int h[MAXN];
    Der(f, n, h), Inv:: Inv(f, n, g);
    Mult(h, g, n, g), Int(g, n, g);
}
namespace Exp {
static int G[MAXN];
void Exp_(int *F, int n) {
    static int H[MAXN];
    if (n == 1) {
        F[0] = 1;
        return;
    }
    Exp_(F, n >> 1);
    Ln(F, n, H);
    Rep(i, n) H[i] = ad(G[i], MOD - H[i]);
    H[0] = ad(H[0], 1);
    Mult(H, F, n, F);
}
void Exp(int *g, int n, int *ans) {
    Rep(i, n) G[i] = g[i], ans[i] = 0;
    Exp_(ans, n);
}
}
void Pow(int *f, int n, int k, int *g) {
    static int h[MAXN];
    Ln(f, n, h);
    Rep(i, n) h[i] = (LL)h[i] * k % MOD;
    Exp:: Exp(h, n, g);
}
namespace Sqrt {
static int A[MAXN], B[MAXN], a[MAXN];
void Sqrt_(int *b, int n) {
    if (n == 1) {
        b[0] = sqrt(a[0]);
        return;
    }
    Sqrt_(b, n >> 1);
    Rep(i, n) A[i] = b[i];
    Mult(A, A, n, A);
    Rep(i, n) A[i] = ad(A[i], a[i]), B[i] = ad(b[i], b[i]);
    Inv:: Inv(B, n, B);
    Mult(A, B, n, b);
}
void Sqrt(int *x, int *y, int n) {
    Rep(i, n) a[i] = x[i], y[i] = 0;
    Sqrt_(y, n);
}
}
int N = Init(131071);
\end{lstlisting}

\subsection{牛顿迭代}
问题:已知$G$,求$F$使得$G(F(x)) = 0$。
已知$F_0$满足$G(F_0(x))\equiv 0 \pmod{x^t}$,则存在:
$$
F(x) \equiv F_0(x) - \frac{G(F_0(x))}{G'(F_0(x))} \pmod {x^{2t}}
$$

其中$G'(F(x)) = \frac{dG}{dF}$

\subsection{MTT}
\begin{lstlisting}
LL MOD;
namespace FFT {
    struct Z {
		LD r, i;
		Z (const LD &r0 = 0, const LD &i0 = 0) : r(r0), i(i0) {}
		Z operator + (const Z& t) const {return Z(r+t.r, i+t.i);}
		Z operator - (const Z& t) const {return Z(r-t.r, i-t.i);}
		Z operator * (const Z& t) const {return Z(r*t.r-i*t.i, r*t.i+i*t.r);}
		Z conj() const {return Z(r, -i);}
		void operator /= (const LD& t) {r /= t, i /= t;}
	};

	int n, bit, rev[MAXN];
	void init(int x) {
		n = 1, bit = 0;
		while(n <= x) n <<= 1, bit++;
		for(int i=1; i<n; i++) rev[i] = (rev[i>>1]>>1) | ((i&1)<<(bit-1));
	}
	void dft(Z *x, int f) {
		for(int i=0; i<n; i++)
			if(i < rev[i])
				swap(x[i], x[rev[i]]);
		for(int w=1; w<n; w<<=1)
		{
			for(int i=0; i<n; i+=(w<<1))
			{
				for(int j=0; j<w; j++)
				{
					Z a = x[i+j], b = x[i+j+w] * Z(cos(PI/w*j), f*sin(PI/w*j));;
					x[i+j] = a + b;
					x[i+j+w] = a - b;
				}
			}
		}
		if(f == -1) for(int i=0; i<n; i++) x[i] /= n;
	}

	Z Xq[MAXN], Yq[MAXN], xlyl[MAXN], xlyh[MAXN], xhyl[MAXN], xhyh[MAXN];

	void mult(LL *x, LL *y, LL *ret) {
		for(int i=0; i<n; i++)
			Xq[i] = Z(x[i]>>15, x[i]&((1<<15)-1)),
				Yq[i] = Z(y[i]>>15, y[i]&((1<<15)-1));
		dft(Xq, +1), dft(Yq, +1);
		for(int i=0; i<n; i++)
		{
			int j = (n-i) & (n-1);
			Z xh = (Xq[i]+Xq[j].conj()) * Z(0.5, 0);
			Z xl = (Xq[i]-Xq[j].conj()) * Z(0, -0.5);
			Z yh = (Yq[i]+Yq[j].conj()) * Z(0.5, 0);
			Z yl = (Yq[i]-Yq[j].conj()) * Z(0, -0.5);
			xhyh[j] = xh*yh, xhyl[j] = xh*yl, xlyh[j] = xl*yh, xlyl[j] = xl*yl;
		}
		for(int i=0; i<n; i++)
			Xq[i] = xhyh[i] + xhyl[i] * Z(0, 1),
				Yq[i] = xlyh[i] + xlyl[i] * Z(0, 1);
		dft(Xq, +1), dft(Yq, +1);
		for(int i=0; i<n; i++)
		{
			LL xhyh = LL(Xq[i].r/n + 0.5) % MOD;
			LL xhyl = LL(Xq[i].i/n + 0.5) % MOD;
			LL xlyh = LL(Yq[i].r/n + 0.5) % MOD;
			LL xlyl = LL(Yq[i].i/n + 0.5) % MOD;
			ret[i] = ((xhyh<<30) + (xhyl<<15) + (xlyh<<15) + (xlyl)) % MOD;
		}
	}
}
//先init,后mult使用即可
\end{lstlisting}

\subsection{FWT}
\begin{lstlisting}
void FWTor(int *a, int ty) {
	for (int i = 2; i <= N; i <<= 1)
		for (int j = 0; j < N; j += i)
			Rep(k, i >> 1)
				if (ty) a[j + (i >> 1) + k] = ad(a[j + (i >> 1) + k], a[j + k]);
				else a[j + (i >> 1) + k] = ad(a[j + (i >> 1) + k], Mod - a[j + k]);
}
void FWTand(int *a, int ty) {
	for (int i = 2; i <= N; i <<= 1)
		for (int j = 0; j < N; j += i)
			Rep(k, i >> 1)
				if (ty) a[j + k] = ad(a[j + k], a[j + (i >> 1) + k]);
				else a[j + k] = ad(a[j + k], Mod - a[j + (i >> 1) + k]);
}
void FWTxor(int *a, int ty) {
	for (int i = 2; i <= N; i <<= 1)
		for (int j = 0; j < N; j += i)
			Rep(k, i >> 1) {
				int x = a[j + k], y = a[j + k + (i >> 1)];
				a[j + k] = ad(x, y), a[j + k + (i >> 1)] = ad(x, Mod - y);
				if (!ty) a[j + k] = a[j + k] * inv2 % Mod, a[j + k + (i >> 1)] = a[j + k + (i >> 1)] * inv2 % Mod;
			}
}
\end{lstlisting}
\subsection{FMT}
\begin{lstlisting}
void FMTor(int *a, int n, int ty) {
	for (int i = 2, p = 1; i <= n; p = i, i <<= 1)
		for (int j = 0; j < n; j += i) Rep(k, p)
			inc(a[j + k + p], (LL)(ty == 1 ? 1 : MOD - 1) * a[j + k] % MOD);
}
void FMTand(int *a, int n, int ty) {
	for (int i = 2, p = 1; i <= n; p = i, i <<= 1)
		for (int j = 0; j < n; j += i) Rep(k, p)
			inc(a[j + k], (LL)(ty == 1 ? 1 : MOD - 1) * a[j + k + p] % MOD);
}
\end{lstlisting}

\section{其它算法}
\subsection{模拟退火}
\begin{lstlisting}
//BZOJ3680
const double eps = 1e-15;
const int maxn = 1005, Times = 15;
int n, w[maxn], x[maxn], y[maxn];
double randdec(double T) { return ((rand() + rand()) - RAND_MAX) * 1. / RAND_MAX * T * 1e-2; }
double calc(double nx, double ny) {
    double sx = 0, sy = 0, len, dx, dy;
    For(i, 1, n) {
        dx = x[i] - nx, dy = y[i] - ny, len = sqrt(dx * dx + dy * dy);
        if (fabs(len) < eps) continue;
        sx += w[i] * dx / len, sy += w[i] * dy / len;
    }
    return sqrt(sx * sx + sy * sy);
}
int main() {
    static double xba, yba, ansx, ansy, tba, ans;
    n = read();
    For(i, 1, n) xba += x[i] = read(), yba += y[i] = read(), w[i] = read();
    ansx = xba /= n, ansy = yba /= n, tba = ans = calc(xba, yba);
    for (int Case = Times; Case --; ) {
        double nowx = xba, nowy = yba, now = tba, res, newx, newy;
        for (double T = 1e6; T >= eps; T *= 0.99) {
            newx = nowx + randdec(T), newy = nowy + randdec(T), res = calc(newx, newy);
            if (res < ans) ans = res, ansx = newx, ansy = newy;
            if (res < now || exp((now - res) / T) * RAND_MAX < rand())
                nowx = newx, nowy = newy, now = res;
        }
    }
    printf("%.3lf %.3lf\n", ansx, ansy);
    return 0;
}
\end{lstlisting}


\section{一些有用的定理和结论}
\begin{itemize}
\item 皮克定理:$2S=2a+b-2$,$a$为内部点数,$b$为边界点数,$S$为面积。
\item 欧拉公式:$F+V=E+C+1$,C表示连通块个数
\end{itemize}

\section{其它代码}
\subsection{pb\_ds的hash\_table}
\begin{lstlisting}
#include<ext/pb_ds/hash_policy.hpp>
#include<ext/pb_ds/assoc_container.hpp>
gp_hash_table<int,bool> h1;
cc_hash_table<int,bool> h2;
\end{lstlisting}
\subsection{\_\_builtin}
\begin{lstlisting}
__builtin_ffs(x)//返回x中最后一个为1的位是从后向前的第几位
__builtin_popcount(x)//x中1的个数。
__builtin_ctz(x)//x末尾0的个数。x=0时结果未定义。
__builtin_clz(x)//x前导0的个数。x=0时结果未定义。
//上面的宏中x都是unsigned int型的,如果传入signed或者是char型,会被强制转换成unsigned int。
__builtin_parity(x)//x中1的奇偶性
\end{lstlisting}
\subsection{std::set}
\begin{lstlisting}
set_union(eg1.begin(),eg1.end(),eg2.begin(),eg2.end(),insert_iterator<set<int> >(eg3,eg3.begin()));
set_intersection(eg1.begin(),eg1.end(),eg2.begin(),eg2.end(),insert_iterator<set<int> >(eg3,eg3.begin()));
set_difference(eg1.begin(),eg1.end(),eg2.begin(),eg2.end(),insert_iterator<set<int> >(eg3,eg3.begin())); //差
set_symmetric_difference(eg1.begin(),eg1.end(),eg2.begin(),eg2.end(),insert_iterator<set<int> >(eg3,eg3.begin())); //对称差
\end{lstlisting}
另外,insert()的返回值为pair<set<TYPE>::iterator, bool>
\subsection{std::bitset}
\begin{lstlisting}
a ^ b //Xor
a & b //And
a | b //Or
bs.any()    //是否存在1
bs.none()    //是否都为0
bs.count()  //1的个数
b.size()     //二进制位的个数
b[pos]       //第 pos 位二进制数
b.test(pos)  //第 pos 位是否为 1
b.set()      //全设为 1
b.set(pos)   //将 pos 处设为 1
b.reset()   //全设为 0
b.reset(pos)   //将 pos 处设为 0
b.flip()     //全部取反
b.flip(pos)  //将 pos 处取反
b.to_ulong() //返回一个 unsigned long 值
b._Find_first() //返回第一个1的位置
b._Find_next(x) //返回x之后下一个1的位置
\end{lstlisting}
\subsection{priority\_queue的重载运算符}
\begin{lstlisting}
struct cmp {
	bool operator()(int x, int y) { return pos[x] > pos[y]; }
};
priority_queue<int, vector<int>, cmp> q;
\end{lstlisting}
\subsection{对拍}
\subsubsection{Windows}
\begin{lstlisting}
@echo off
set /a i=1
:loop
echo Case %i%:
set /a i=i+1
gen.exe
a.exe
bf.exe
fc a.out a.ans
if not errorlevel 1 goto loop
pause
\end{lstlisting}
\subsubsection{Linux}
\begin{lstlisting}
#!/bin/bash
for i in $(seq 1 100000);do
    ./gen
    ./a
    ./a1
    if diff a.out a1.out; then
    	echo $i "AC"
    else
        echo $i "WA"
        exit 0
    fi
done
\end{lstlisting}
\subsection{编译选项}
\begin{lstlisting}
-fsanitize=address,undefined
\end{lstlisting}

%==============================正文部分==============================%
\end{document}
